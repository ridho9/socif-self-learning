\documentclass[conference]{IEEEtran}

\usepackage{cite}
\usepackage{listings}
\usepackage{hyperref}
\usepackage{color}
\usepackage{amsmath}
% \usepackage{biblatex}

\lstset{
    language=[LaTeX]TeX,
    keepspaces,
    captionpos=b,
    breaklines=true,
    frame=tb,
}

\title{Learning \LaTeX}
\author{
    \IEEEauthorblockN{Ridho Pratama}
    \IEEEauthorblockA{
        Teknik Informatika\\
        Sekolah Teknik Elektro dan Informatika\\
        Institut Teknologi Bandung\\
        Bandung, Indonesia\\
        Email: p.ridho@students.itb.ac.id\\
    }
}
\date{28 April 2019}

\begin{document}
\maketitle

\begin{abstract}
    While using WYSIWYG ("what you see is what you get") text editor,
    like Microsoft Word or Google Docs,
    have you ever tried to format something to what you want, and
    getting frustated because the result is not what you want, while the editor
    also messing up the rest of your works? Me too.

    But there are better alternatives to write documents professionally.
    One of the alternatives is \LaTeX.
    With this paper I would like to learn and use \LaTeX{} to write this paper,
    and share the experiences in learning to use \LaTeX.
\end{abstract}

\section{Introduction}
    LaTeX (stylized as \LaTeX), is a document preparation system as an improvement
    on TeX system created by Donald Knuth.
    When writing using LaTeX, the writer uses plain text instead of formatted text found in WYSIWYG text
    editor like Microsoft Word. The writer uses markup tagging to define the 
    structure of a document, to stylize text in a document.
    The writer usually writes the document in a .tex file, and then use program
    to compile the document into output file (such as PDF or DVI).

    Documents produced using LaTeX usually are of higher quality than documents
    produced using WYSIWYG editor.
    This is because when using LaTeX, the writer only focuses on the contents
    without caring about the formatting like margins, font and fontsize,
    styling, etc. Those are handled by LaTeX, so the writer could get high-quality
    and consistent document.

    LaTeX is widely used in academia for publication of scientific documents
    in many fields.

\section{Learning \LaTeX}
\subsection{Installation}
    Steps to install LaTeX depends on your operating system.

    If you use Windows, you could use MiKTeX (\url{https://miktex.org}),
    which is a Windows distribution of
    LaTeX that includes additional packages and a TeX editor called TeXworks.

    On my system which uses Linux, specifically Arch Linux, where usually 
    TeX Live is used.
    To install TeX Live in Arch Linux, you just run \texttt{pacman -S texlive-most}
    which already includes the LaTeX binary and additional packages.

    You could write .tex document in any plain text editor, even notepad.
    But for ease, I use Visual Studio Code (\url{https://code.visualstudio.com/})
    with LaTeX Workshop extension installed.
    The extension adds features that helps you writing .tex document, like
    auto-completion, auto-compile, and live document preview.

\subsection{Your first document}
    It's time to create your first latex document.
    Create a new file named \texttt{first.tex}, and fill it with:

    \lstinputlisting[label=lst1, caption=first.tex]{lst/lst1.tex}

    Then open terminal in your current folder and run \texttt{pdflatex first.tex}
    to compile the document into a PDF file named \texttt{first.pdf}.
    Open the PDF file and you should file the text `This is my first latex document'
    in it, and congratulations, you have created your first latex document.

\subsection{Document class}
    You should see that the document in previous subsection already have a lot
    of things set up for you,
    like margin, font and font size, spacing, etc, and you only have to write the
    text you wanted.
    That part is set up with \lstinline!\documentclass{article}! part.
    That command lets you specify to the compiler that you want to create an 
    article. There are many built-in document classes already specified for
    various uses. Class `article' is for articles in scientific journals,
    presentations, etc, `book' is for real book, `letter' is for creating letters,
    `beamer' for writing presentations, and many other for more specific uses.

\subsection{Document environment}
    The next part is
    \lstinline!\begin{document}...\end{document}!
    That pair of command specify an `environment' in which the item inside it have
    some rules applied to it. In the example we used `document' environment,
    which tells the compiler that it is our document and then everything inside 
    the environment will be written in output document.
    This also means, any normal text outside of the document environment will
    not be printed. In fact, we will get an error when compiling if we have normal
    text outside of document environment. 

    There are many other environment for other specific uses,
    like `description', `list', `enumerate' and `itemize' to create
    various numbered and bulleted lists, `figure' and `table' to
    create figures and tables, `math' and `displaymath' to type math,
    `array' to create array, `equation' to type equations, and many
    other more. We also could create our own custom enviroment or
    use enviroments create by other people by using packages.

\subsection{Line and paragraph}
    Create file \texttt{second.tex} and type:
    \lstinputlisting[label=second, caption=second.tex]{lst/lst2.tex}
    And compile the file. There is many other thing in the listing,
    but we will focus on section 1.1.

    You could see that how the text are typed in the .tex file
    didn't affect how the paragraph are being splitted in the
    result file. In latex, texts are usually typed in paragraph.
    In the text file, you separate between paragraph using blank lines.
    And so, the lines without blank lines between in will be treated
    as one paragraph in the result. LaTeX will automatically add
    separating spaces.

\subsection{Titles and author}
    Back to listing \ref{second}, the code in line 3-5 specifies
    the documents title, author, and written date. 
    But this only specify the attributes and doesn't create any text
    in output.
    To put the title and author in the output file, we use
    \lstinline!\maketitle! to print it.

\subsection{Document structure}
    In listing \ref{second}, There are also one more thing, that is
    the section and subsection. Those are the document structure marker,
    using that we could structure our document into sections and subsections.
    The section numberings are automatic.
    Using the document structure could also help us in generating automatic
    table of contents.

    This is specifically for `article' document class.
    For other classes, the structuring might be different, like `book' class
    has chapter structure, etc.

\subsection{Packages}
    LaTeX and TeX is different. As stated in the Introduction, LaTeX is an improvement
    from TeX. One of the improvement is ability to uses packages. Packages is just
    like in many programming languages, where we could import functions and add
    additional functionalities.

    Importing packages in LaTeX is done with \lstinline!\usepackage{packagename}! command,
    and usually you put the command between \lstinline!\documentclass{}!
    and \lstinline!\begin{document}!.

\subsection{Math}
    In many academic papers, we want to insert math equations to the document.
    There are many ways we could insert equations that fit our needs.

    First is inline equation, which is put inline with the text, one of the example
    of the result is $f(x) = x^2$, which is typed like \lstinline!$f(x) = x^2$!.

    Second is displayed equation, in which the equation have it own line.
    Example: `\lstinline!\[ 1 + 2 = 3 \]!'
    Which results in:
    \[ 1 + 2 = 3 \]

    Third is align environment, in which you use `align' and `align*' environment
    to write multiple equation that is aligned.
    We need import `amsmath' package to get this enviroment.
    Example:
    \begin{lstlisting}[label=alignmath, caption={Align environment example}]
\begin{align*}
1 + 2 &= 3\\
4 &= 3 + 1
\end{align*}
    \end{lstlisting}
    Which results in:
    \begin{align*}
    1 + 2 &= 3\\
    4 &= 3 + 1
    \end{align*}
    `align*' enviroment will align the equation on the `\&' mark,
    and the `\textbackslash \textbackslash' is used to split the equations.
    `align' (notice the lack of `*') will also insert numbering for
    each line of equation.

\subsection{Code listing}
    Listings are code snippets like listing \ref{lst1} and \ref{second}.
    You could insert code listing in your document by using the `listings' package,
    and then writing the code inside a `lstlisting' environment.

    You could customize the appearance, the language, font, and many others,
    and you could customize them individually for each listing enviroment,
    or you could set them globally with \lstinline!\lstset{}! in the beginning
    of the document before you start the document environment.

\section{Conclusion}
    \LaTeX is a document preparation which is used to create high quality documents.
    In LaTeX, unlike MS Word, you type your document in plain text with added commands,
    and then compile your document to the desired output format.

    LaTeX has various commands that you can use to create various type of documents,
    from scientific paper, letters, books, even presentations. You could also extends
    the functionality that LaTeX has by using packages.

\bibliographystyle{IEEEtran}
\bibliography{main}
\nocite{vellage_2017, ieeetran, latexwb}

\end{document}
