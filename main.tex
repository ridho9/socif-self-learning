\documentclass[conference]{IEEEtran}

\usepackage{cite}
\usepackage{listings}

% \bibliographystyle{IEEEtran}
% \bibliography{main}

\title{Learning \LaTeX}
\author{
    \IEEEauthorblockN{Ridho Pratama}
    \IEEEauthorblockA{
        Teknik Informatika\\
        Sekolah Teknik Elektro dan Informatika\\
        Institut Teknologi Bandung\\
        Bandung, Indonesia\\
        Email: p.ridho@students.itb.ac.id\\
    }
}

\begin{document}
\maketitle

\begin{abstract}
    While using WYSIWYG ("what you see is what you get") text editor,
    like Microsoft Word or Google Docs,
    have you ever tried to format something to what you want, and
    getting frustated because the result is not what you want, while the editor
    also messing up the rest of your works? Me too.

    But there are better alternatives to write documents professionally.
    One of the alternatives is \LaTeX.
    With this paper I would like to learn and use \LaTeX to write this paper,
    and share the experiences in learning to use \LaTeX.
\end{abstract}

\section{Introduction}
    LaTeX (stylized: \LaTeX), is a document preparation system. When writing,
    the writer uses plain text instead of formatted text found in WYSIWYG text
    editor like Microsoft Word. The writer uses markup tagging to define the 
    structure of a document, to stylize text in a document.
    The writer usually writes the document in a .tex file, and then use program
    to compile the document into output file (such as PDF or DVI).

    Documents produced using LaTeX usually are of higher quality than documents
    produced using WYSIWYG editor.
    This is because when using LaTeX, the writer only focuses on the contents
    without caring about the formatting like margins, font and fontsize,
    styling, etc. Those are handled by LaTeX, so the writer could get high-quality
    and consistent document.

    LaTeX is widely used in academia for publication of scientific documents
    in many fields.


\end{document}
